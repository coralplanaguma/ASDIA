\documentclass[]{article}
\usepackage{lmodern}
\usepackage{amssymb,amsmath}
\usepackage{ifxetex,ifluatex}
\usepackage{fixltx2e} % provides \textsubscript
\ifnum 0\ifxetex 1\fi\ifluatex 1\fi=0 % if pdftex
  \usepackage[T1]{fontenc}
  \usepackage[utf8]{inputenc}
\else % if luatex or xelatex
  \ifxetex
    \usepackage{mathspec}
  \else
    \usepackage{fontspec}
  \fi
  \defaultfontfeatures{Ligatures=TeX,Scale=MatchLowercase}
\fi
% use upquote if available, for straight quotes in verbatim environments
\IfFileExists{upquote.sty}{\usepackage{upquote}}{}
% use microtype if available
\IfFileExists{microtype.sty}{%
\usepackage{microtype}
\UseMicrotypeSet[protrusion]{basicmath} % disable protrusion for tt fonts
}{}
\usepackage[margin=1in]{geometry}
\usepackage{hyperref}
\hypersetup{unicode=true,
            pdftitle={ASDIA Projet 1},
            pdfborder={0 0 0},
            breaklinks=true}
\urlstyle{same}  % don't use monospace font for urls
\usepackage{color}
\usepackage{fancyvrb}
\newcommand{\VerbBar}{|}
\newcommand{\VERB}{\Verb[commandchars=\\\{\}]}
\DefineVerbatimEnvironment{Highlighting}{Verbatim}{commandchars=\\\{\}}
% Add ',fontsize=\small' for more characters per line
\usepackage{framed}
\definecolor{shadecolor}{RGB}{248,248,248}
\newenvironment{Shaded}{\begin{snugshade}}{\end{snugshade}}
\newcommand{\KeywordTok}[1]{\textcolor[rgb]{0.13,0.29,0.53}{\textbf{#1}}}
\newcommand{\DataTypeTok}[1]{\textcolor[rgb]{0.13,0.29,0.53}{#1}}
\newcommand{\DecValTok}[1]{\textcolor[rgb]{0.00,0.00,0.81}{#1}}
\newcommand{\BaseNTok}[1]{\textcolor[rgb]{0.00,0.00,0.81}{#1}}
\newcommand{\FloatTok}[1]{\textcolor[rgb]{0.00,0.00,0.81}{#1}}
\newcommand{\ConstantTok}[1]{\textcolor[rgb]{0.00,0.00,0.00}{#1}}
\newcommand{\CharTok}[1]{\textcolor[rgb]{0.31,0.60,0.02}{#1}}
\newcommand{\SpecialCharTok}[1]{\textcolor[rgb]{0.00,0.00,0.00}{#1}}
\newcommand{\StringTok}[1]{\textcolor[rgb]{0.31,0.60,0.02}{#1}}
\newcommand{\VerbatimStringTok}[1]{\textcolor[rgb]{0.31,0.60,0.02}{#1}}
\newcommand{\SpecialStringTok}[1]{\textcolor[rgb]{0.31,0.60,0.02}{#1}}
\newcommand{\ImportTok}[1]{#1}
\newcommand{\CommentTok}[1]{\textcolor[rgb]{0.56,0.35,0.01}{\textit{#1}}}
\newcommand{\DocumentationTok}[1]{\textcolor[rgb]{0.56,0.35,0.01}{\textbf{\textit{#1}}}}
\newcommand{\AnnotationTok}[1]{\textcolor[rgb]{0.56,0.35,0.01}{\textbf{\textit{#1}}}}
\newcommand{\CommentVarTok}[1]{\textcolor[rgb]{0.56,0.35,0.01}{\textbf{\textit{#1}}}}
\newcommand{\OtherTok}[1]{\textcolor[rgb]{0.56,0.35,0.01}{#1}}
\newcommand{\FunctionTok}[1]{\textcolor[rgb]{0.00,0.00,0.00}{#1}}
\newcommand{\VariableTok}[1]{\textcolor[rgb]{0.00,0.00,0.00}{#1}}
\newcommand{\ControlFlowTok}[1]{\textcolor[rgb]{0.13,0.29,0.53}{\textbf{#1}}}
\newcommand{\OperatorTok}[1]{\textcolor[rgb]{0.81,0.36,0.00}{\textbf{#1}}}
\newcommand{\BuiltInTok}[1]{#1}
\newcommand{\ExtensionTok}[1]{#1}
\newcommand{\PreprocessorTok}[1]{\textcolor[rgb]{0.56,0.35,0.01}{\textit{#1}}}
\newcommand{\AttributeTok}[1]{\textcolor[rgb]{0.77,0.63,0.00}{#1}}
\newcommand{\RegionMarkerTok}[1]{#1}
\newcommand{\InformationTok}[1]{\textcolor[rgb]{0.56,0.35,0.01}{\textbf{\textit{#1}}}}
\newcommand{\WarningTok}[1]{\textcolor[rgb]{0.56,0.35,0.01}{\textbf{\textit{#1}}}}
\newcommand{\AlertTok}[1]{\textcolor[rgb]{0.94,0.16,0.16}{#1}}
\newcommand{\ErrorTok}[1]{\textcolor[rgb]{0.64,0.00,0.00}{\textbf{#1}}}
\newcommand{\NormalTok}[1]{#1}
\usepackage{graphicx,grffile}
\makeatletter
\def\maxwidth{\ifdim\Gin@nat@width>\linewidth\linewidth\else\Gin@nat@width\fi}
\def\maxheight{\ifdim\Gin@nat@height>\textheight\textheight\else\Gin@nat@height\fi}
\makeatother
% Scale images if necessary, so that they will not overflow the page
% margins by default, and it is still possible to overwrite the defaults
% using explicit options in \includegraphics[width, height, ...]{}
\setkeys{Gin}{width=\maxwidth,height=\maxheight,keepaspectratio}
\IfFileExists{parskip.sty}{%
\usepackage{parskip}
}{% else
\setlength{\parindent}{0pt}
\setlength{\parskip}{6pt plus 2pt minus 1pt}
}
\setlength{\emergencystretch}{3em}  % prevent overfull lines
\providecommand{\tightlist}{%
  \setlength{\itemsep}{0pt}\setlength{\parskip}{0pt}}
\setcounter{secnumdepth}{0}
% Redefines (sub)paragraphs to behave more like sections
\ifx\paragraph\undefined\else
\let\oldparagraph\paragraph
\renewcommand{\paragraph}[1]{\oldparagraph{#1}\mbox{}}
\fi
\ifx\subparagraph\undefined\else
\let\oldsubparagraph\subparagraph
\renewcommand{\subparagraph}[1]{\oldsubparagraph{#1}\mbox{}}
\fi

%%% Use protect on footnotes to avoid problems with footnotes in titles
\let\rmarkdownfootnote\footnote%
\def\footnote{\protect\rmarkdownfootnote}

%%% Change title format to be more compact
\usepackage{titling}

% Create subtitle command for use in maketitle
\newcommand{\subtitle}[1]{
  \posttitle{
    \begin{center}\large#1\end{center}
    }
}

\setlength{\droptitle}{-2em}

  \title{ASDIA Projet 1}
    \pretitle{\vspace{\droptitle}\centering\huge}
  \posttitle{\par}
    \author{}
    \preauthor{}\postauthor{}
    \date{}
    \predate{}\postdate{}
  

\begin{document}
\maketitle

\section{Préliminaires}\label{preliminaires}

\section{Partie 1 : Apprentissage supervisé pour la
classification}\label{partie-1-apprentissage-supervise-pour-la-classification}

\subsection{Objectifs de l'apprentissage
supervisé}\label{objectifs-de-lapprentissage-supervise}

On suppose qu'une catégorie binaire, \(y \in \{0,1\}\), est associée un
vecteur d'observations multidimensionelles, notées
\({\bf x} \in \mathbb{R}^D\). Par exemple, \({\bf x}\) peut représenter
un image, un texte, un fichier audio, etc. La classification peut
indiquer la présence d'un animal domestique (chat vs chien), une
appréciation subjective (sentiment positif ou négatif), un style musical
(rock ou pop), etc. Dans tous ces exemples, on aura spécifié la
representation vectorielle des observations d'une manière ou d'une
autre, par une méthode d'encodage spécifique à l'application considérée.

L'objectif de l'apprentissage supervisé est de construire un modèle de
prédiction probabiliste, \(q({\bf x}, \theta) \in [0,1]\), évaluant,
pour tout \({\bf x}\), la probabilité pour que l'observation \({\bf x}\)
provienne de la catégorie 1. Pour cela, la méthode adoptée consiste à
minimiser une fonction de perte moyenne

\[
  L(\theta) = \mathbb{E}[L(q({\bf x}, \theta), y)] \, .
\]

Les coefficients du modèle, \(\theta\), sont appelés \emph{paramètres}
ou \emph{poids}. Le problème se résume à minimiser la fonction de perte
moyenne exprimée en fonction de \(\theta\), c'est à dire calculer les
coefficients suivants

\[
\hat \theta = \arg\min_{\theta} L(\theta) \, .
\]

D'autres coefficients que les paramètres du modèle peuvent aussi entrer
en compte dans le processus de minimisation. Ces coefficients
supplémentaires sont appelés les \emph{hyperparamètres}. Ils peuvent
représenter le nombre d'itérations de l'algorithme de minimisation, la
dimension ou la complexité du modèle (nombre de neurones ou de couches
de neurones par exemple), des paramètres de régularisation, la condition
initiale de l'algorithme de minimisation (seed du générateur), la
méthode numérique choisie pour réaliser la minimisation de
\(L(\theta)\), etc.

La valeur théorique de l'espérance étant difficile à calculer, la perte
moyenne est généralement évaluée à partir d'un échantillon de données
d'apprentissage \(({\bf x}_i, y_i)\), \(i = 1, \dots, n\). Cet ensemble
de données est généralement de grande taille et il permet (sous des
hypothèses raisonnable) d'approcher la valeur moyenne de la fonction de
perte de la manière suivante

\[
   L(\theta) \approx \frac1n \sum_{i = 1}^n L(q({\bf x}_i, \theta), y_i) \, .
\]

La fonction de perte empirique définie ci-dessus est alors utilisée pour
calculer \(\hat \theta\). Un ensemble de test est généralement utilisé
pour évaluer les performances prédictives du modèle et choisir un
ensemble d'hyperparamètres conduisant à des performances satisfaisantes.

\subsection{Perte log-loss (entropie
croisée)}\label{perte-log-loss-entropie-croisee}

La fonction de perte \emph{log-loss} est très fréquemment considérée par
les algorithmes d'apprentissage probabiliste. Elle est définie de la
manière suivante

\[
L(q({\bf x}), y) = - y \log q({\bf x}) - (1-y)\log (1 - q({\bf x})) \, ,
\] où \(q({\bf x})\) est identifiée à la probabilité d'observer la
classe 1 sachant \({\bf x}\), \(q({\bf x}) = q_1({\bf x})\).

La perte moyenne associée à la fonction de perte \textbf{log-loss} est
égale à

\[
\mathbb{E}[L(q({\bf x}), y )] =  \int \mathbb{E}[L(q({\bf x}), y) | {\bf x}] p({\bf x}) d{\bf x} \, .
\] En termes de théorie de l'information, minimiser la perte log-loss
moyenne revient à minimiser l'information perdue lorsque l'on approche
la probabilité cible \(p(y = 1|{\bf x})\) par un prédicteur probabiliste
donné, \(q({\bf x})\).

L'objet de cette partie est de démontrer le résultat suivant.

\textbf{Théorème.} \emph{Soit \(q({\bf x})\) une fonction arbitraire à
valeur dans \([0,1]\) et} \[
q_{\rm opt}({\bf x}) = p(y = 1|{\bf x}) \, , \quad \forall {\bf x}.
\] \emph{Alors, nous avons} \[
\mathbb{E}[L(q({\bf x}), y )] \geq \mathbb{E}[L(q_{\rm opt}({\bf x}), y )] .
\]

\begin{itemize}
\item
  \textbf{Conditionnement} : Montrer que pour minimiser l'espérance
  \(\mathbb{E}[L(q({\bf x}), y )]\), il suffit de minimiser l'espérance
  conditionnelle \(\mathbb{E}[L(q({\bf x}), y ) | {\bf x}]\) pour tout
  \({\bf x}\).
\item
  \textbf{Entropie croisée} : On définit l'\emph{entropie croisée} de
  deux lois de Bernoulli \(p\) et \(p^\prime\) (attribuant des valeurs à
  0 ou 1 uniquement) de la manière suivante
\end{itemize}

\[
h(p,p^\prime) = - p(y = 0) \log p^\prime(y = 0) - p(y = 1) \log p^\prime(y = 1) .
\] Démontrer que

\[
h(p,p^\prime) = \mathbb{KL}(p \| p^\prime) + h(p)  \geq h(p) \, , 
\] où \(\mathbb{KL}(p \| p^\prime)\) est la divergence Kullback-Leibler
et \(h(p)\) l'entropie de \(p\).

\[
\begin{align}
\mathbb{KL}(p \| p^\prime) &= p(y = 0) \log \frac{p(y = 0)}{p \prime y = 0} + p(y =1) \log \frac{p(y=1)}{p \prime (y = 1)}\\
&= h(p, p\prime) + p(y=0)\log p(y=0) + p(y=1) \log p(y=1)\\
&= h(p, p \prime) - h(p)
\end{align}
\]

Comme la divergence de Kulback-Leibler est positive, on obtient bien
l'inégalité attendue.

\begin{itemize}
\tightlist
\item
  \textbf{Entropie croisée et log-loss } : On appelle \(q_{.}({\bf x})\)
  la loi de Bernoulli de paramètre \(q({\bf x})\). Démontrer que
\end{itemize}

\[
\mathbb{E}[L(q({\bf x}), y ) | {\bf x}]  = h(p_{\sf Y}(.|{\bf x}), q_{.}({\bf x}))  .  
\]

\begin{itemize}
\tightlist
\item
  Démontrer le théorème.
\end{itemize}

En conclusion, minimiser l'espérance conditionnelle
\(\mathbb{E}[L(q({\bf x}), y) | {\bf x} ]\) revient à minimiser la
divergence \(D_{KL}( p_{\sf y|{\bf x}} \| q({\bf x}) )\). Ainsi, nous
pouvons affirmer que minimiser la perte moyenne log-loss revient à
minimiser l'information perdue lorsque l'on approche la loi cible
\(p(y|{\bf x})\) par un modèle probabiliste \(q({\bf x})\).

\begin{itemize}
\tightlist
\item
  On suppose que la perte moyenne est évaluée à partir d'un échantillon
  de données d'apprentissage \(({\bf x}_i, y_i)\), \(i = 1, \dots, n\),
  indépendantes et de même loi. On suppose que la loi conditionnelle est
  modélisée à l'aide d'ensemble de paramètres, \(\theta\), par la
  fonction suivante
\end{itemize}

\[
p(y = 1 | {\bf x}, \theta) = q({\bf x}, \theta) \, .
\] Démontrer que le minimum de la fonction de perte correspond au
maximum de la vraisemblance pour ce modèle.

\subsection{Réseaux de neurones}\label{reseaux-de-neurones}

\subsubsection{Neurone probabiliste}\label{neurone-probabiliste}

Soit \({\bf x} \in \mathbb{R}^D\). On appelle \emph{neurone
probabiliste} un prédicteur probabiliste de la forme suivante

\[
 q({\bf x}, \theta) =  \sigma( {\bf w}^T {\bf x}  - b ) \, , \quad  {\bf x} \in \mathbb{R}^D,
\] où la fonction sigmoide (\(\sigma\)) est définie par

\[
\sigma(x) = \frac{1}{1 + e^{-x}}\, , \quad  x \in \mathbb{R},
\] et le paramètre \(\theta\) possède \((D+1)\) coefficients

\[
\theta = (b, {\bf w}).
\]

Le vecteur de coefficients \({\bf w}\) est de dimension \(D\) et \(b\)
est un scalaire représentant un seuil.

La motivation initiale d'un modèle neuronal est d'imiter un neurone
biologique. En effet, les variables \({\bf x}\) représentent des stimuli
externes, les coefficients \({\bf w}\) représentent des poids
synaptiques, et la fonction ``sigmoide'' correspond à la fonction
d'activation d'un neurone biologique. Le neurone emet un potentiel
d'action avec une probabilité d'autant plus grande que la somme pondérée
des stimuli est grande.

La notation peut être simplifiée en ajoutant une variable d'entrée
\(x_{0}\) fixée de manière constante à la valeur \(-1\)
(\(x_{0} = -1\)). Dans ce cas, nous pouvons augmenter la dimension du
vecteur \({\bf w}\) à \(D+1\), et prendre \(w_0 = b\). La fonction de
decision neuronale est alors décrite par la formule suivante

\[
 q({\bf x}, {\bf w}) =  \sigma( {\bf w}^T {\bf x} ) \, , \quad  {\bf x} \in \mathbb{R}^D.
\] Dans cette notation (abusive), nous n'avons pas augmenté la dimension
de \({\bf x}\) à \(D+1\) pour bien mettre en évidence que l'espace des
observation est resté identique.

Notons que la fonction sigmoide n'est pas la seule fonction d'activation
proposée dans les réseaux neuronaux. La fonction \emph{rectified linear
unit} (\emph{relu}) est souvent utilisée dans les couches cachées pour
son efficacité numérique. La fonction \emph{relu} est définie par

\[
{\rm relu}(x) = \max(0, x) , \, \quad x \in \mathbb{R}.
\]

Elle est parfois remplacée par la fonction \emph{softplus} définie de la
manière suivante

\[
{\rm softplus}(x) = \log(1 + e^{x}), \, \quad x \in \mathbb{R}.
\] Lorsque l'on encode les variables de classes \(y\) en variables
catégorielles, \(y = 1\) est codé par (0,1) et \(y = 0\) est codé par
\((1,0)\) on remplace la fonction sigmoide par la fonction softmax.
Cette dernière peut être généralisée à plusieurs catégories. Cela a un
intêret s'il y a plus de deux catégories.

\subsubsection{Réseau de neurones
probabilistes}\label{reseau-de-neurones-probabilistes}

L'intelligence artificielle propose d'approcher le prédicteur optimal
par des combinaisons de fonctions neuronales agencées en couches
successives, comme dans le cerveau humain. L'idée n'est pas d'utiliser
une base fonctionnelle particulière mais d'apprendre une base adaptable
au problème considéré, afin d'utiliser les meilleures représentations
des variables observées dans chaque problème particulier.

Par exemple, un réseau de neurones possédant une couche cachée s'écrit
de la manière suivante

\[
 q({\bf x}, {\bf \theta}) = \sigma({\bf W_0}^T \, \sigma({\bf W_1}^T{\bf x}) ) \, .
\]

où le paramètre global, \(\theta\), inclut les seuils d'activation et
les coefficients synaptiques. Il correspond à la formule suivante \[
\theta = ({\bf W_0}, {\bf W_1}).
\]

Les notations \({\bf W_0}^T\) et \({\bf W_1}^T\) désignent désormais des
matrices de dimensions respectives \((L\times 1 + 1)\) et
\((D \times L + L)\) et la fonction \emph{sigmoide} est appliquée à
chaque coordonnée des vecteurs. Le paramètre \(L\) correspond au nombre
de neurones dans la couche cachée. Le paramètre global \(\theta\) est
donc de dimension \((D + 2)L + 1\). Ainsi, pour deux dimensions
(\(D =2\)) et 20 neurones, nous aurons 81 paramètres distincts dans le
modèle.

La formule peut se généraliser à l'utilisation de plusieurs couches de
neurones. On parle alors de réseaux multi-couches. On qualifie les
réseaux multi-couches de \emph{réseaux profonds} à partir d'une douzaine
de couches (hum deux selon certains auteurs). Par exemple, un réseau de
neurones possédant 3 couches cachées s'écrit

\[
 q({\bf x}, {\bf \theta}) = \sigma( {\bf W_0}^T \sigma({\bf W_1}^T \sigma({\bf W_2}^T \sigma({\bf W_3}^T{\bf x})))) \, , \quad {\bf x} \in \mathbb{R}^D.
\] {[}Un réseau de neurones artificiels. Le réseau prend une variable de
dimension 3 en entrée et la classe en deux catégories (softmax) en
sortie. Il y a une couche cachée possédant 4 neurones.{]}

\subsubsection{Gradient stochastique}\label{gradient-stochastique}

L'apprentissage, c'est à dire la minimisation effective de la fonction
de perte \(L(\theta)\), est généralement réalisé par une méthode
d'optimisation numérique. Il existe de nombreuses méthodes
d'optimisation numérique permettant de trouver un minimum local de la
fonction \(L(\theta)\).

La méthode de prédilection des réseaux de neurones est la méthode du
gradient. Il s'agit d'une méthode de minimisation locale. La méthode du
gradient repose sur le calcul des dérivées partielles de la fonction
\(L(\theta)\)

\[
\nabla L = \left( \frac{\partial L}{\partial \theta_1} , \dots, \frac{\partial L}{\partial \theta_m} \right)^T \,.
\]

D'après la formule de Taylor (ou d'après la définition de l'application
différentielle), la fonction \(L(\theta)\) peut être développée
localement de la manière suivante

\[
L(\theta + d\theta) = L(\theta) + {\bf \nabla} L(\theta)^T d\theta + o(d\theta) \, .
\] Ainsi, les déplacements \(d\theta\) conduisant aux plus grandes
variations de la fonction \(L(\theta)\) au voisinage de \(\theta\)
s'effectuent dans une direction parallèle au vecteur
\(\nabla L(\theta)\).

La méthode de gradient est une méthode rudimentaire cherchant à suivre
la ligne de plus grande pente afin de minimiser la fonction de perte,
\(L(\theta)\). Il s'agit d'une méthode itérative qui met à jour le
paramètre \(\theta\) de la manière suivante

\[
\theta^{t+1} = \theta^t - \eta \nabla L(\theta^t) \, .
\]

La constante \(\eta\) est petite et elle est appelée \emph{taux
d'apprentissage} (learning rate). Lors d'une mise à jour de
l'algorithme, la valeur de la fonction \(L\) a ainsi tendance à
décroitre

\[
L(\theta^{t+1}) - L(\theta^t) \approx - \eta \| {\bf \nabla} L(\theta^t) \|_2^2 \,  < 0 \,,
\] et \(\theta^t\) converge vers un minimum local, un point selle ou un
plateau de \(L(\theta)\).

Cette méthode peut être raffinée en utilisant les dérivées secondes de
la fonction \(L(\theta)\). Ceci est pertinent lorsque le paramètre
\(\theta\) est de dimension raisonnable, de l'ordre de quelques dizaines
de coefficients. En grande dimension, la méthode du du gradient reste la
méthode favorite dans les implémentations logicielles des réseaux
neuronaux.

Les méthodes de \emph{gradient stochastique} sont utilisées en très
grande dimension. Elles implémentent l'algorithme suivant

\begin{enumerate}
\def\labelenumi{\arabic{enumi}.}
\tightlist
\item
  Choisir une valeur initiale \(\theta\) et une valeur du taux
  d'apprentissage \(\eta\).
\item
  Répéter les étapes suivantes jusqu'à ce qu'un minimum acceptable soit
  atteint
\end{enumerate}

\begin{itemize}
\tightlist
\item
  permuter aléatoirement les éléments de l'échantillon d'apprentissage
\item
  pour tout \(i\) de 1 à \(n\), mettre à jour les paramètres de la
  manière suivante \[
  \theta^{t+1} = \theta^{t} - \eta_t  \nabla L(q({\bf x}_i , \theta^t), y_i) 
  \] Un tel cycle aléatoire s'appelle une \emph{époque}.
\end{itemize}

La phase d'initialisation est une phase cruciale dans les applications.
Pour la reconnaissance des images, de nombreux réseaux de neurones
pré-entrainés existent et atteignent d'excellentes performances de
classification une base de données publiques appelée ``Imagenet''. Les
logiciels actuels permettent de récupérer les matrices de coefficients
de réseaux pré-entrainés, simplifiant considérablement la phase
d'initialisation d'un réseau de neurones destiné à reconnaître des
images particulières, proche de celles annotées dans la base. On peut,
en quelques heures, entraîner un réseau de neurones à reconnaître des
animaux domestiques à partir de quelques milliers d'images. Construire
et entraîner un tel réseau sans cette condition initiale peut nécessiter
des ressources considérables.

Plutôt que de parcourir les données de l'échantillon les unes après les
autres (point 2 de l'algorithme), un compromis entre l'algorithme de
gradient et l'algorithme de gradient stochastique consiste à considérer
des paquets (``batch'') de données lors de la mise à jour des
paramètres. Cette technique est plus rapide mais un peu moins précise
que le gradient stochastique. La taille du batch devient un
hyperparamètre supplémentaire sur lequel agir lors de la phase
d'apprentissage.

L'algorithme de gradient stochastique s'accompagne de conditions
concernant le taux d'apprentissage. Ce paramètre de l'algorithme peut
décroître avec le temps. Une condition classique consiste à choisir
\(\eta_t\) tel que \[
\sum_{t=1}^\infty \eta_t = \infty \quad {\rm et } \quad \sum_{t=1}^\infty \eta_t^2 < \infty.
\] De nombreuses variantes d'algorithmes d'optimisation sont disponibles
dans les programmes de réseaux de neurones. Il est souvent utile de
pouvoir en tester plusieurs.

\section{Partie 2 : Reconnaissance de chiffres
manuscrits}\label{partie-2-reconnaissance-de-chiffres-manuscrits}

Dans cet exercice, nous déterminerons les capacités des réseaux de
neurones de la bibliothèque \emph{Tensorflow} à correctement reconnaître
des chiffres écrits à la main. La reconnaissance de l'écriture
manuscrite est un problème difficile, et constitue un test pour les
algorithmes d'apprentissage. Cet exemple est historique pour les réseaux
de neurones et largement traité sur internet (il vous sera donc facile
de savoir si vous faites bien).

Les données de caractères manuscrits sont issues de la base
\href{https://fr.wikipedia.org/wiki/Base_de_donn\%C3\%A9es_MNIST}{MNIST}
(Modified National Institute of Standards and Technology). La base de
données MNIST est une base de données de chiffres écrits à la main. Elle
regroupe 60000 images de chiffres manuscrits et 10000 images de test. Ce
sont des matrices carrées ayant 28 pixels de côté, dont les valeurs sont
des niveaux de gris allant de 0 à 255.

\subsection{Lecture des données}\label{lecture-des-donnees}

Les données sont accessibles depuis la librairie R \texttt{keras} qui
les télécharge directement depuis l'url de dépot (il faut donc être
connecté à internet pour réaliser le TP). Chaque donnée est représentée
par un tableau 3d (\texttt{array}) contenant les informations (images,
width, height) en niveaux de gris. Pour coder le vecteur \({\bf x}\),
nous aurons besoin de vectoriser ces données. Pour ne pas se limiter aux
tutoriaux disponibles sur le web, nous nous intéressons à la
reconnaissance des chiffres \(2\) et 7.

\begin{itemize}
\tightlist
\item
  Lire les données MNIST. Il y a 4 ensembles de variables.
\end{itemize}

\begin{Shaded}
\begin{Highlighting}[]
  \KeywordTok{library}\NormalTok{(keras)}
  \KeywordTok{library}\NormalTok{(magrittr)}
\NormalTok{  mnist <-}\StringTok{ }\KeywordTok{dataset_mnist}\NormalTok{()}
\NormalTok{  x_train <-}\StringTok{ }\NormalTok{mnist}\OperatorTok{$}\NormalTok{train}\OperatorTok{$}\NormalTok{x}
\NormalTok{  y_train <-}\StringTok{ }\NormalTok{mnist}\OperatorTok{$}\NormalTok{train}\OperatorTok{$}\NormalTok{y}
\NormalTok{  x_test <-}\StringTok{ }\NormalTok{mnist}\OperatorTok{$}\NormalTok{test}\OperatorTok{$}\NormalTok{x}
\NormalTok{  y_test <-}\StringTok{ }\NormalTok{mnist}\OperatorTok{$}\NormalTok{test}\OperatorTok{$}\NormalTok{y}
\end{Highlighting}
\end{Shaded}

\begin{itemize}
\tightlist
\item
  Filtrer les données correspondant aux chiffres \(2\) et 7. Commenter
  et compléter le code suivant. Exécuter le code en changeant l'option
  de chunck.
\end{itemize}

\begin{Shaded}
\begin{Highlighting}[]
\CommentTok{# comment 1}
\NormalTok{  boo_train <-}\StringTok{ }\NormalTok{y_train }\OperatorTok{==}\StringTok{ }\DecValTok{2} \OperatorTok{|}\StringTok{ }\NormalTok{y_train }\OperatorTok{==}\StringTok{ }\DecValTok{7} 
\NormalTok{  x_train <-}\StringTok{ }\NormalTok{mnist}\OperatorTok{$}\NormalTok{train}\OperatorTok{$}\NormalTok{x[boo_train,,]}
\NormalTok{  y_train <-}\StringTok{ }\NormalTok{mnist}\OperatorTok{$}\NormalTok{train}\OperatorTok{$}\NormalTok{y[boo_train]}

\CommentTok{# comment 2}
\NormalTok{  boo_test <-}\StringTok{ }\NormalTok{y_test }\OperatorTok{==}\StringTok{ }\DecValTok{2} \OperatorTok{|}\StringTok{ }\NormalTok{y_test }\OperatorTok{==}\StringTok{ }\DecValTok{7}
\NormalTok{  x_test <-}\StringTok{ }\NormalTok{mnist}\OperatorTok{$}\NormalTok{test}\OperatorTok{$}\NormalTok{x[boo_test,,]}
\NormalTok{  y_test <-}\StringTok{ }\NormalTok{mnist}\OperatorTok{$}\NormalTok{test}\OperatorTok{$}\NormalTok{y[boo_test]}
\end{Highlighting}
\end{Shaded}

\begin{itemize}
\tightlist
\item
  Utiliser la fonction \texttt{image()} pour visualiser le premier
  chiffre test de la base de données réduite. C'est un 7.
\end{itemize}

\begin{Shaded}
\begin{Highlighting}[]
  \KeywordTok{image}\NormalTok{(}\KeywordTok{t}\NormalTok{(x_test[}\DecValTok{1}\NormalTok{, }\DecValTok{28}\OperatorTok{:}\DecValTok{1}\NormalTok{,]), }\DataTypeTok{col =} \KeywordTok{grey.colors}\NormalTok{(}\DecValTok{5}\NormalTok{))}
\end{Highlighting}
\end{Shaded}

\includegraphics{ASDIA_Projet_1_files/figure-latex/unnamed-chunk-3-1.pdf}

\begin{itemize}
\tightlist
\item
  Les images de dimension 28x28 doivent être converties en vecteurs de
  longueur 784 (\(= 28 \times 28\)). Cela peut se faire de plusieurs
  manières. En particulier, la fonction \texttt{array\_reshape()} de
  keras est très utile pour cela.
\end{itemize}

\begin{Shaded}
\begin{Highlighting}[]
\CommentTok{# reshape}
\NormalTok{  x_train <-}\StringTok{ }\KeywordTok{array_reshape}\NormalTok{(x_train, }\KeywordTok{c}\NormalTok{(}\KeywordTok{nrow}\NormalTok{(x_train), }\DecValTok{784}\NormalTok{))}
\NormalTok{  x_test <-}\StringTok{ }\KeywordTok{array_reshape}\NormalTok{(x_test, }\KeywordTok{c}\NormalTok{(}\KeywordTok{nrow}\NormalTok{(x_test), }\DecValTok{784}\NormalTok{))}
\end{Highlighting}
\end{Shaded}

\begin{itemize}
\tightlist
\item
  Normaliser les données pour obtenir des valeurs réelles (flottants)
  entre 0 et 1 en divisant les valeurs présentes par 255. Utiliser la
  fonction \texttt{image()} pour visualiser le premier chiffre test de
  la base de données réduite dans cette nouvelle représentation.
\end{itemize}

\begin{Shaded}
\begin{Highlighting}[]
\CommentTok{# rescale}
\NormalTok{  x_train <-}\StringTok{ }\NormalTok{x_train}\OperatorTok{/}\DecValTok{255}
\NormalTok{  x_test <-}\StringTok{ }\NormalTok{x_test}\OperatorTok{/}\DecValTok{255}

\CommentTok{# le symbol %>% est similaire au 'pipe' d'unix (library(magrittr))}
  \KeywordTok{dim}\NormalTok{(x_test)}
\NormalTok{  x_test[}\DecValTok{1}\NormalTok{,] }\OperatorTok\StringTok{ }\KeywordTok{matrix}\NormalTok{(}\DataTypeTok{nrow =} \DecValTok{28}\NormalTok{) }\OperatorTok\StringTok{ }\NormalTok{.[,}\DecValTok{28}\OperatorTok{:}\DecValTok{1}\NormalTok{] }\OperatorTok\StringTok{ }\KeywordTok{image}\NormalTok{(}\DataTypeTok{col =} \KeywordTok{grey.colors}\NormalTok{(}\DecValTok{5}\NormalTok{))}
\end{Highlighting}
\end{Shaded}

\begin{itemize}
\tightlist
\item
  Les données de classe sont des entiers 2,7. Convertir ces données en
  variables booléennes ou binaires. La valeur 1 ou \texttt{TRUE}
  correspondra à un 7.
\end{itemize}

\begin{Shaded}
\begin{Highlighting}[]
\NormalTok{  y_train <-}\StringTok{ }\NormalTok{y_train }\OperatorTok{==}\StringTok{ }\DecValTok{7}
\NormalTok{  y_test <-}\StringTok{  }\NormalTok{y_test }\OperatorTok{==}\StringTok{ }\DecValTok{7}
\end{Highlighting}
\end{Shaded}

\subsection{Ajuster un réseau de neurone avec
keras}\label{ajuster-un-reseau-de-neurone-avec-keras}

\begin{itemize}
\item
  Construire un réseau de neurones à deux couches cachées à l'aide de la
  fonction \texttt{keras\_model\_sequential()}, en faisant varier les
  paramètres des couches comme ci-dessous. Définir les termes
  apparaissant dans la construction du modèle.
\item
  Commenter et compléter le code suivant.
\end{itemize}

\begin{Shaded}
\begin{Highlighting}[]
\CommentTok{# commenter les lignes suivantes}
\NormalTok{  model <-}\StringTok{ }\KeywordTok{keras_model_sequential}\NormalTok{() }
\NormalTok{  model }\OperatorTok\StringTok{ }
\StringTok{    }\KeywordTok{layer_dense}\NormalTok{(}\DataTypeTok{units =} \DecValTok{256}\NormalTok{, }\DataTypeTok{activation =} \StringTok{'relu'}\NormalTok{, }\DataTypeTok{input_shape =} \DecValTok{784}\NormalTok{) }\OperatorTok\StringTok{ }
\StringTok{    }\KeywordTok{layer_dropout}\NormalTok{(}\DataTypeTok{rate =} \FloatTok{0.5}\NormalTok{) }\OperatorTok\StringTok{ }
\StringTok{    }\KeywordTok{layer_dense}\NormalTok{(}\DataTypeTok{units =} \DecValTok{128}\NormalTok{, }\DataTypeTok{activation =} \StringTok{'relu'}\NormalTok{) }\OperatorTok
\StringTok{    }\KeywordTok{layer_dropout}\NormalTok{(}\DataTypeTok{rate =} \FloatTok{0.5}\NormalTok{) }\OperatorTok
\StringTok{    }\KeywordTok{layer_dense}\NormalTok{(}\DataTypeTok{units =} \DecValTok{1}\NormalTok{, }\DataTypeTok{activation =} \StringTok{'sigmoid'}\NormalTok{)}
\end{Highlighting}
\end{Shaded}

\begin{itemize}
\tightlist
\item
  Donner un tableau de correspondance entre les notions mathématiques
  introduites dans la section précédente et les termes apparaissant dans
  la construction du modèle.
\end{itemize}

\begin{Shaded}
\begin{Highlighting}[]
\NormalTok{terme <-}\StringTok{ }\KeywordTok{c}\NormalTok{(}\StringTok{"units = 256"}\NormalTok{, }\StringTok{"input_shape"}\NormalTok{)}
\NormalTok{definition <-}\StringTok{ }\KeywordTok{c}\NormalTok{(}\StringTok{"Dimension de la première couche cachée (W_2)"}\NormalTok{, }\StringTok{"A compléter"}\NormalTok{)}
\KeywordTok{data.frame}\NormalTok{(terme, definition)}
\end{Highlighting}
\end{Shaded}

\begin{verbatim}
##         terme                                   definition
## 1 units = 256 Dimension de la première couche cachée (W_2)
## 2 input_shape                                  A compléter
\end{verbatim}

\begin{itemize}
\tightlist
\item
  Compiler le modèle en précisant la fonction de perte (binary) et
  demander de visualiser le taux de bonne classification (accuracy). Le
  choix de l'optimiseur est spécifié.
\end{itemize}

\begin{Shaded}
\begin{Highlighting}[]
\CommentTok{# commenter et compléter les lignes suivantes  }
\NormalTok{  model }\OperatorTok\StringTok{ }\KeywordTok{compile}\NormalTok{(}
    \DataTypeTok{loss =} \StringTok{'also_sprach_zarathustra'}\NormalTok{,}
    \DataTypeTok{optimizer =} \KeywordTok{optimizer_rmsprop}\NormalTok{(}\DataTypeTok{lr =} \FloatTok{0.001}\NormalTok{, }\DataTypeTok{rho =}\NormalTok{ O, }\DataTypeTok{decay =} \DecValTok{0}\NormalTok{),}
    \DataTypeTok{metrics =} \KeywordTok{c}\NormalTok{(}\StringTok{'ex_nihilo_nihil'}\NormalTok{)}
\NormalTok{)}
\end{Highlighting}
\end{Shaded}

\begin{itemize}
\tightlist
\item
  Donner un tableau de correspondance entre les notions mathématiques
  introduites dans la section précédente et les termes apparaissant dans
  la compilation du modèle.
\end{itemize}

\begin{Shaded}
\begin{Highlighting}[]
\NormalTok{terme <-}\StringTok{ }\KeywordTok{c}\NormalTok{(}\StringTok{"loss"}\NormalTok{, }\StringTok{"optimizer"}\NormalTok{, }\StringTok{"lr"}\NormalTok{, }\StringTok{"rho"}\NormalTok{, }\StringTok{"decay"}\NormalTok{, }\StringTok{"accuracy"}\NormalTok{)}
\NormalTok{definition <-}\StringTok{ }\KeywordTok{c}\NormalTok{(}\StringTok{"A compléter"}\NormalTok{,}\StringTok{""}\NormalTok{,}\StringTok{""}\NormalTok{,}\StringTok{""}\NormalTok{,}\StringTok{""}\NormalTok{,}\StringTok{""}\NormalTok{)}
\KeywordTok{data.frame}\NormalTok{(terme, definition)}
\end{Highlighting}
\end{Shaded}

\begin{verbatim}
##       terme  definition
## 1      loss A compléter
## 2 optimizer            
## 3        lr            
## 4       rho            
## 5     decay            
## 6  accuracy
\end{verbatim}

\begin{itemize}
\tightlist
\item
  Allons-y Alonso, pour ajuster le réseau (apprentissage)
\end{itemize}

\begin{Shaded}
\begin{Highlighting}[]
\NormalTok{  history <-}\StringTok{ }\NormalTok{carpe_diem }\OperatorTok\StringTok{ }\KeywordTok{fit}\NormalTok{(}
\NormalTok{                        x_train, }
\NormalTok{                        y_train, }
                        \DataTypeTok{epochs =} \DecValTok{20}\NormalTok{, }
                        \DataTypeTok{batch_size =} \DecValTok{128}\NormalTok{, }
                        \DataTypeTok{validation_data =} \KeywordTok{list}\NormalTok{(x_test, y_test)}
\NormalTok{)}
\end{Highlighting}
\end{Shaded}

\begin{itemize}
\item
  Donner un tableau de correspondance entre les notions mathématiques
  introduites dans la section précédente et les termes apparaissant dans
  l'apprentissage du modèle.
\item
  Quelle est signification des courbes \texttt{loss} et
  \texttt{val\_loss} que l'on voit tracées.
\end{itemize}

\begin{Shaded}
\begin{Highlighting}[]
\KeywordTok{plot}\NormalTok{(history)}
\end{Highlighting}
\end{Shaded}

\begin{itemize}
\tightlist
\item
  Evaluer le modèle sur les données de test (erreur de classification et
  perte log loss)
\end{itemize}

\begin{Shaded}
\begin{Highlighting}[]
\NormalTok{model }\OperatorTok\StringTok{ }\KeywordTok{ex_nihilo_nihil}\NormalTok{(x_test, y_test)}
\end{Highlighting}
\end{Shaded}

\begin{itemize}
\tightlist
\item
  Donner une matrice de confusion pour les classements effectués par le
  modèle sur le jeu test
\end{itemize}

\begin{Shaded}
\begin{Highlighting}[]
\CommentTok{# help(table)}
\NormalTok{pred_class <-}\StringTok{ }\NormalTok{model }\OperatorTok\StringTok{ }\KeywordTok{predict_classes}\NormalTok{(x_test)}
\KeywordTok{quod_errat_demonstratum}\NormalTok{(}\DataTypeTok{predicted =}\NormalTok{ pred_class, }\DataTypeTok{observed =}\NormalTok{ mnist}\OperatorTok{$}\NormalTok{test}\OperatorTok{$}\NormalTok{y[boo_test])}
\end{Highlighting}
\end{Shaded}

\begin{itemize}
\tightlist
\item
  Montrer deux chiffres manuscrits que la machine n'a pas réussi à
  classer correctement. Donner la liste des mal classés et les
  probabilités de classement pour chacune des erreurs.
\end{itemize}

\subsection{Défi MNIST}\label{defi-mnist}

Pour les données de l'exercice précédent, répondre aux questions
suivantes.

\begin{itemize}
\tightlist
\item
  Reporter dans un tableau 6x2 (\texttt{data.frame}), puis dans deux
  diagrammes en barres (\texttt{barplot}), les valeurs des erreurs de
  classification et de \emph{logloss} obtenues sur l'ensemble test pour
  6 réseaux de neurones distincts dont on aura fait varier les
  paramètres de la manière suivante
\item
  nombre de couches cachées : 1 ou 5,
\item
  nombre de neurones par couche cachée : 10 ou 100,
\item
  valeur de \texttt{dropout} par couche cachée : 0.2 ou 0.8.
\end{itemize}

Inclure dans tableau une colonne indiquant le nom de chaque modele. Par
exemple, un modèle à 5 couches cachées, 10 neurones par couche et une
valeur ``dropout'' de 0.2 pourra s'appeler ``modele\_5\_10\_2''

\begin{Shaded}
\begin{Highlighting}[]
\KeywordTok{paste}\NormalTok{(}\KeywordTok{c}\NormalTok{(}\StringTok{"modele"}\NormalTok{, }\DecValTok{5}\NormalTok{, }\DecValTok{10}\NormalTok{, }\DecValTok{2}\NormalTok{), }\DataTypeTok{collapse =} \StringTok{"_"}\NormalTok{)}
\end{Highlighting}
\end{Shaded}

\begin{verbatim}
## [1] "modele_5_10_2"
\end{verbatim}

\begin{itemize}
\item
  Quel modèle de prédiction vous parait être le meilleur ?
\item
  Donner la liste des chiffres mal classés et les probabilités de
  classement pour chacune des erreurs.
\end{itemize}

\section{Partie 3 : Analyse de critiques de films
(IMDB)}\label{partie-3-analyse-de-critiques-de-films-imdb}

L'objectif de cette séance est de répondre à un défi de classification
portant sur l'analyse d'opinion à partir de documents textuels
(``natural language processing'').

Les documents analysés sont des critiques de films écrites par des
utilisateurs du site web ``Internet Movie Data Base'' (IMDB). À chaque
critique est associée une note donnée par l'utilisateur du site. Seules
les notes extrêmes ont été conservées et converties en valeurs binaires
représentant des opinions positives ou négatives envers le film.

La base de données comporte 50000 critiques de films. Chaque document
est prétraité et représenté sous le format d'un \emph{sac de mots}
(``bag of words'') pour en faciliter l'analyse. Un sac de mots peut être
vu comme une représentation des termes d'un document partir des
fréquences d'occurrence des mots dans la base IMDB.

Dans cette séance de TP, nous téléchargerons un échantillon de la base
de données et constituerons un sous-échantillon comportant 10000
documents annotés. Chaque document est associé à une évaluation
traduisant une opinion positive (valeur 0) ou négative (valeur 1) des
utilisateurs.

Le but de ce TP est de prédire le mieux possible l'opinion ou le
\emph{sentiment} des utilisateurs à partir des fréquences d'occurrence
de certains termes apparaissant dans les textes. Un échantillon de test
comportant 5000 critiques indépendantes sera utilisé par l'enseignant
pour évaluer la méthode sélectionnée dans le défi final.

\begin{Shaded}
\begin{Highlighting}[]
\KeywordTok{library}\NormalTok{(magrittr)}
\KeywordTok{library}\NormalTok{(keras)}
\end{Highlighting}
\end{Shaded}

\subsection{Index des termes et sacs de
mots.}\label{index-des-termes-et-sacs-de-mots.}

À chaque terme d'un texte (aussi appelé \emph{document}) est associé une
fréquence d'apparition globale dans la base de données IMDB. Les termes
sont référencés dans un index à l'aide d'un nombre indiquant leur rang
d'apparition dans la base de données. Dans l'index, les termes sont
triés du plus fréquent au moins fréquent.

L'index peut être consulté à partir de la bibliothèque \texttt{keras},
sous forme de liste dont les attributs sont les termes utilisés dans la
base de données. Pour cela, on utlise la fonction
\texttt{dataset\_imdb\_word\_index()}. L'index est illustré ci-dessous.

Le numéro trouvé dans l'index correspond au rang d'un terme donné. Pour
trouver le mot le plus fréquemment utilisé dans l'IMDB, on peut chercher
le terme dont la valeur est égale au rang 1. L'index contient aussi
quatres valeurs spéciales (start, pad, unknown, unused).

\begin{Shaded}
\begin{Highlighting}[]
\NormalTok{  index <-}\StringTok{ }\NormalTok{keras}\OperatorTok{::}\KeywordTok{dataset_imdb_word_index}\NormalTok{()}
  \KeywordTok{names}\NormalTok{(index[index }\OperatorTok{==}\StringTok{ }\DecValTok{1}\NormalTok{])}
\end{Highlighting}
\end{Shaded}

\begin{verbatim}
## [1] "the"
\end{verbatim}

Sans surprise, le terme le plus fréquemment utilisé est l'article
``the''. Sa valeur dans l'index est donc égale à 1.

L'article ``the'' se retrouve à la position 85976 dans l'index. L'index
n'est donc pas ordonné par la fréquence des termes, mais par une
permutation. Nous pouvons utiliser la fonction \texttt{order()} pour
réordonner l'index et trouver les 10 termes les plus utilisés dans
l'ensemble des critiques de films.

\begin{Shaded}
\begin{Highlighting}[]
\NormalTok{  o <-}\StringTok{ }\KeywordTok{as.numeric}\NormalTok{(index) }\OperatorTok\StringTok{ }\KeywordTok{order}\NormalTok{()}
\NormalTok{  index[o[}\DecValTok{1}\OperatorTok{:}\DecValTok{10}\NormalTok{]] }\OperatorTok\StringTok{ }\KeywordTok{names}\NormalTok{()}
\end{Highlighting}
\end{Shaded}

\begin{verbatim}
##  [1] "the" "and" "a"   "of"  "to"  "is"  "br"  "in"  "it"  "i"
\end{verbatim}

Nous voyons qu'il s'agit d'articles, de prépositions ou de termes
non-informatifs, comme par exemple, des éléments extraits des balises
html (br). Il sera peut-être préférable d'éliminer les termes les plus
utilisés. L'entrée 49 correspondant au terme ``good''.

Pour réduire le temps de calcul, nous conservons uniquement les 5000
termes les plus fréquents. Dans la suite, chaque document sera
représenté par une suite de fréquences de termes représentés par les
codes de l'index. Le jeu de données résultant de cette étape est nommé
\texttt{imbd}.

\begin{Shaded}
\begin{Highlighting}[]
\NormalTok{  imbd <-}\StringTok{  }\NormalTok{keras}\OperatorTok{::}\KeywordTok{dataset_imdb}\NormalTok{(}\DataTypeTok{path =} \StringTok{"imdb.npz"}\NormalTok{,}
                              \DataTypeTok{num_words =} \DecValTok{5048}\NormalTok{,}
                              \DataTypeTok{skip_top =} \DecValTok{49}\NormalTok{,}
                              \DataTypeTok{oov_char =} \OperatorTok{-}\DecValTok{2}\NormalTok{,}
                              \DataTypeTok{start_char =} \OperatorTok{-}\DecValTok{1}\NormalTok{,}
                              \DataTypeTok{index_from =} \DecValTok{0}\NormalTok{)}
\end{Highlighting}
\end{Shaded}

Les données enregistrées dans l'objet \texttt{imbd} se présentent sous
la forme de listes de documents (train et test). On utilisera les
``double-crochets'' ou le symbole dollar pour accéder aux attributs de
ces listes.

\begin{Shaded}
\begin{Highlighting}[]
  \KeywordTok{summary}\NormalTok{(imbd)}
\end{Highlighting}
\end{Shaded}

\begin{verbatim}
##       Length Class  Mode
## train 2      -none- list
## test  2      -none- list
\end{verbatim}

Par exemple le document numéro 12 dans l'ensemble d'apprentissage est lu
de la manière suivante.

\begin{Shaded}
\begin{Highlighting}[]
\NormalTok{  document <-}\StringTok{ }\NormalTok{imbd}\OperatorTok{$}\NormalTok{train}\OperatorTok{$}\NormalTok{x[[}\DecValTok{12}\NormalTok{]]}
\NormalTok{  document}
\end{Highlighting}
\end{Shaded}

\begin{verbatim}
##  [1]   -2   51   -2 1607   -2   -2   -2   66   52  361 1395   -2   51   -2
## [15]  216   -2   -2 1703   -2   -2   -2   -2  326   -2  173  326   71   -2
## [29]   -2  870   -2  153   68   75   -2   -2  319   -2   -2   -2   -2  246
## [43]   -2   62   -2   -2  376   -2   97  154   -2   -2  907   -2  546   -2
## [57]   -2 1493   -2   -2   -2   -2   -2   -2  209   -2   -2   -2 1319  988
## [71]   -2 2999   -2  422   -2   70 2215  546   -2   -2  152   -2   97  760
## [85]  376   -2  100  348   -2   -2  199   -2 2238   -2   -2  317   -2   -2
## [99]  454
\end{verbatim}

Nous voyons que le terme d'indice -2 est fréquent. Après filtrage, il
correspond en fait à la chaîne de caractères \textbf{UNK} signifiant
unknown. Cette chaîne apparait car nous nous sommes restreint à un
dictionnaire de 5000 mots.

Les fréquences des termes apparaissant dans le document 12 peuvent être
données par la fonction \texttt{table()}. Cette fonction utilise l'ordre
alphabétique pour représenter les comptages de chaque terme.

\begin{Shaded}
\begin{Highlighting}[]
  \KeywordTok{table}\NormalTok{(}\KeywordTok{sapply}\NormalTok{(document[document }\OperatorTok{>}\StringTok{ }\OperatorTok{-}\StringTok{ }\DecValTok{2}\NormalTok{], }\DataTypeTok{FUN =} \ControlFlowTok{function}\NormalTok{(x) }\KeywordTok{names}\NormalTok{(index[index }\OperatorTok{==}\StringTok{ }\NormalTok{x])))}
\end{Highlighting}
\end{Shaded}

\begin{verbatim}
## 
##            0            5       actors          add        after 
##            1            1            1            1            1 
##          bad       chosen       comedy        could         dead 
##            1            1            1            2            1 
##       disney       except expectations     expected         give 
##            1            2            1            1            1 
##          had         half    laughable         less          lot 
##            1            1            1            2            1 
##          low  masterpiece      murders     realized       rented 
##            1            1            1            1            1 
##          saw        story       stupid    stupidity         than 
##            1            1            2            1            1 
##        thing        title         very         well         were 
##            1            1            1            1            1 
##         when         wife         work        worst 
##            2            1            1            1
\end{verbatim}

Le document 12 contient des termes plutôt négatifs. Il est associé à une
opinion négative (valeur 0). Cela se vérifie en affichant la variable
\(y\).

\begin{Shaded}
\begin{Highlighting}[]
\NormalTok{  imbd}\OperatorTok{$}\NormalTok{train}\OperatorTok{$}\NormalTok{y[}\DecValTok{12}\NormalTok{]}
\end{Highlighting}
\end{Shaded}

\begin{verbatim}
## [1] 0
\end{verbatim}

Pour la suite, c'est à vous de jouer. Pour ne pas entrer dans les
subtilités du traitement du langage, nous nous appuierons sur un codage
simplifié des textes basé sur les fréquences des mots de l'index
(one-hot encoding). Pour un encodage par `embeddings', une méthode
particulièrement dédiée à l'analyse de texte, on peut trouver des
solutions sur le web.

\subsection{\texorpdfstring{Défi ``analyse de
sentiments''}{Défi analyse de sentiments}}\label{defi-analyse-de-sentiments}

\subsubsection{Lecture des données}\label{lecture-des-donnees-1}

\begin{itemize}
\tightlist
\item
  Ecrire une ligne de commande R permettant convertir le document 12 en
  un vecteur de longueur 5000, indiquant le nombre d'apparition de
  chacun des indices allant de 1 à 5000 dans ce document.
\end{itemize}

\begin{Shaded}
\begin{Highlighting}[]
  \CommentTok{# comment 1}
  \KeywordTok{help}\NormalTok{(sapply)}
\NormalTok{  result <-}\StringTok{ }\KeywordTok{sapply}\NormalTok{(}\DecValTok{49}\OperatorTok{:}\DecValTok{5048}\NormalTok{, }\DataTypeTok{FUN =} \ControlFlowTok{function}\NormalTok{(x) cogito_ergo_sum)}
\NormalTok{  result[}\DecValTok{1}\OperatorTok{:}\DecValTok{100}\NormalTok{]}
\end{Highlighting}
\end{Shaded}

\begin{itemize}
\tightlist
\item
  Constituer un jeu de données comportant 10000 documents choisis pour
  moitié dans l'ensemble ``train'' et pour moitié dans l'ensemble
  ``test'' de l'IMBD. Techniquement nous le constituerons en 20 étapes,
  pour limiter l'impact sur la mémoire. Commenter et exécuter le code
  suivant
\end{itemize}

\begin{Shaded}
\begin{Highlighting}[]
\NormalTok{x_imbd <-}\StringTok{ }\OtherTok{NULL}

  \ControlFlowTok{for}\NormalTok{ (i }\ControlFlowTok{in} \DecValTok{1}\OperatorTok{:}\DecValTok{10}\NormalTok{)\{}
    
\NormalTok{    x_imbd_}\DecValTok{500}\NormalTok{ <-}\StringTok{ }\OtherTok{NULL}
    
      \ControlFlowTok{for}\NormalTok{ (j }\ControlFlowTok{in}\NormalTok{ (}\DecValTok{500}\OperatorTok{*}\NormalTok{(i}\OperatorTok{-}\DecValTok{1}\NormalTok{)}\OperatorTok{+}\DecValTok{1}\NormalTok{)}\OperatorTok{:}\NormalTok{(}\DecValTok{500}\OperatorTok{*}\NormalTok{i))\{}
        
        \CommentTok{# comment 1}
        
\NormalTok{        doc_temp <-}\StringTok{ }\NormalTok{imbd}\OperatorTok{$}\NormalTok{train}\OperatorTok{$}\NormalTok{x[[j]]}
\NormalTok{        x_imbd_}\DecValTok{500}\NormalTok{ <-}\StringTok{ }\KeywordTok{rbind}\NormalTok{(x_imbd_}\DecValTok{500}\NormalTok{, }
                         \KeywordTok{sapply}\NormalTok{(}\DecValTok{49}\OperatorTok{:}\DecValTok{5048}\NormalTok{, }
                                \DataTypeTok{FUN =} \ControlFlowTok{function}\NormalTok{(ind) }\KeywordTok{sum}\NormalTok{(doc_temp }\OperatorTok{==}\StringTok{ }\NormalTok{ind)))}
        
        \ControlFlowTok{if}\NormalTok{ (j}\OperatorTok\DecValTok{500} \OperatorTok{==}\StringTok{ }\DecValTok{0}\NormalTok{) }\KeywordTok{print}\NormalTok{(j) }\CommentTok{# ca rassure}
\NormalTok{    \}}
\NormalTok{    x_imbd <-}\StringTok{ }\KeywordTok{rbind}\NormalTok{(x_imbd, x_imbd_}\DecValTok{500}\NormalTok{)}
\NormalTok{  \}}

 \ControlFlowTok{for}\NormalTok{ (i }\ControlFlowTok{in} \DecValTok{1}\OperatorTok{:}\DecValTok{10}\NormalTok{)\{}
   
\NormalTok{    x_imbd_}\DecValTok{500}\NormalTok{ <-}\StringTok{ }\OtherTok{NULL}
    
      \ControlFlowTok{for}\NormalTok{ (j }\ControlFlowTok{in}\NormalTok{ (}\DecValTok{500}\OperatorTok{*}\NormalTok{(i}\OperatorTok{-}\DecValTok{1}\NormalTok{)}\OperatorTok{+}\DecValTok{1}\NormalTok{)}\OperatorTok{:}\NormalTok{(}\DecValTok{500}\OperatorTok{*}\NormalTok{i))\{}
        
        \CommentTok{# comment 2}
        
\NormalTok{        doc_temp <-}\StringTok{ }\NormalTok{imbd}\OperatorTok{$}\NormalTok{test}\OperatorTok{$}\NormalTok{x[[j]]}
\NormalTok{        x_imbd_}\DecValTok{500}\NormalTok{ <-}\StringTok{ }\KeywordTok{rbind}\NormalTok{(x_imbd_}\DecValTok{500}\NormalTok{, }
                         \KeywordTok{sapply}\NormalTok{(}\DecValTok{49}\OperatorTok{:}\DecValTok{5048}\NormalTok{, }
                                \DataTypeTok{FUN =} \ControlFlowTok{function}\NormalTok{(ind) }\KeywordTok{sum}\NormalTok{(doc_temp }\OperatorTok{==}\StringTok{ }\NormalTok{ind)))}
        
        \ControlFlowTok{if}\NormalTok{ (j}\OperatorTok\DecValTok{500} \OperatorTok{==}\StringTok{ }\DecValTok{0}\NormalTok{) }\KeywordTok{print}\NormalTok{(j) }\CommentTok{# ca rassure}
\NormalTok{    \}}
\NormalTok{    x_imbd <-}\StringTok{ }\KeywordTok{rbind}\NormalTok{(x_imbd, x_imbd_}\DecValTok{500}\NormalTok{)}
\NormalTok{  \}}
\end{Highlighting}
\end{Shaded}

\begin{itemize}
\item
  Que contient l'objet \texttt{x\_imbd} ?
\item
  Définir les classes \(y = 0\) ou \(y=1\) pour chaque élément de
  \texttt{x\_imbd}
\end{itemize}

\begin{Shaded}
\begin{Highlighting}[]
\NormalTok{  y_imbd <-}\StringTok{ }\KeywordTok{c}\NormalTok{(imbd}\OperatorTok{$}\NormalTok{train}\OperatorTok{$}\NormalTok{y[}\DecValTok{1}\OperatorTok{:}\DecValTok{5000}\NormalTok{], imbd}\OperatorTok{$}\NormalTok{test}\OperatorTok{$}\NormalTok{y[}\DecValTok{1}\OperatorTok{:}\DecValTok{5000}\NormalTok{])}
\end{Highlighting}
\end{Shaded}

Et voilà. On est en pleine forme et on dispose d'une base
d'apprentissage comportant les fréquences d'apparition des mots de
l'index pour 10000 documents (\texttt{x\_imbd}) et les opinions des
utilisateurs \texttt{y\_imbd}. Le défi peut vraiment commencer.

\paragraph{Etude d'association}\label{etude-dassociation}

Le but d'une étude d'association est d'identifier les termes les plus
associés aux opinions positives ou négatives des utilisateurs. Pour
cela, nous evaluons la correlation au carré entre l'occurrence de chaque
terme et l'opinion de l'utilisateur (présence d'un 1). Il se peut que
certaines valeurs de corrélation ne soient pas calculables à cause d'un
écart-type nul.

\begin{itemize}
\tightlist
\item
  Calculer le coefficient de corrélation au carré entre les fréquences
  d'apparition des termes de l'index et opinion des utilsateurs (5000
  valeurs).
\end{itemize}

\begin{Shaded}
\begin{Highlighting}[]
  \CommentTok{# comment}
\NormalTok{  x <-}\StringTok{ }\NormalTok{x_imbd}
  
  \CommentTok{# comment}
\NormalTok{  y <-}\StringTok{ }\NormalTok{y_imbd}
  
  \CommentTok{# comment}
\NormalTok{  r <-}\StringTok{ }\NormalTok{find_me}
\NormalTok{  r2 <-}\StringTok{ }\NormalTok{r}\OperatorTok{^}\DecValTok{2}
\end{Highlighting}
\end{Shaded}

\begin{itemize}
\tightlist
\item
  Montrer les termes dont la valeur d'association \(r^2\) est supérieure
  à 3 pour cent (0.02), puis supérieure à 0.02, et à 0.01. \emph{Note} :
  Il faut effectuer un décalage de 48 indices dans l'index pour trouver
  le codage correct.
\end{itemize}

\begin{Shaded}
\begin{Highlighting}[]
  \CommentTok{# quelque chose à changer}
\NormalTok{  index[o[ }\KeywordTok{which}\NormalTok{(r2 }\OperatorTok{>}\StringTok{ }\FloatTok{0.02}\NormalTok{) }\OperatorTok{+}\StringTok{ }\DecValTok{48}\NormalTok{ ]] }\OperatorTok\StringTok{ }\KeywordTok{names}\NormalTok{()}
\end{Highlighting}
\end{Shaded}

\begin{itemize}
\tightlist
\item
  Dans quelles proportions les termes de valeur d'association \(r^2\)
  supérieure à 0.02 apparaissent-ils dans les documents ? Représenter
  graphiquement ces proportions à l'aide d'un diagramme en barre.
\end{itemize}

\begin{Shaded}
\begin{Highlighting}[]
\CommentTok{# Calculer la frequence des termes realisant la condition}
\NormalTok{  freq <-}\StringTok{ }\NormalTok{x[, conditio_logicae ] }\OperatorTok\StringTok{ }\KeywordTok{apply}\NormalTok{(}\DecValTok{2}\NormalTok{, mean) }

\CommentTok{# mots dans l'index et barplot}
  \KeywordTok{names}\NormalTok{(freq) <-}\StringTok{  }\NormalTok{index[o[}\KeywordTok{which}\NormalTok{(r2 }\OperatorTok{>}\StringTok{ }\FloatTok{0.02}\NormalTok{) }\OperatorTok{+}\StringTok{ }\DecValTok{48}\NormalTok{]] }\OperatorTok\StringTok{ }\KeywordTok{names}\NormalTok{() }
  \KeywordTok{barplot}\NormalTok{(}\KeywordTok{sort}\NormalTok{(freq, }\DataTypeTok{decreasing =} \OtherTok{TRUE}\NormalTok{), }\DataTypeTok{col =} \StringTok{"lightblue"}\NormalTok{, }\DataTypeTok{las =} \DecValTok{2}\NormalTok{)}
\end{Highlighting}
\end{Shaded}

\begin{itemize}
\tightlist
\item
  Dans quelles proportions les termes de valeur d'association \(r^2\)
  supérieure à 0.02 apparaissent-ils dans les documents ? Représenter
  graphiquement ces proportions à l'aide d'un diagramme en barre.
\end{itemize}

\begin{Shaded}
\begin{Highlighting}[]
\NormalTok{  yes_we_can}
\end{Highlighting}
\end{Shaded}

\begin{itemize}
\tightlist
\item
  Dans quelles proportions les termes de valeur d'association
  \(r = cor(x,y)\) supérieure à 0.1 apparaissent-ils dans les documents
  ? Représenter graphiquement ces proportions à l'aide d'un diagramme en
  barre.
\end{itemize}

\begin{Shaded}
\begin{Highlighting}[]
\NormalTok{  yes_we_can}
  \KeywordTok{barplot}\NormalTok{(}\KeywordTok{sort}\NormalTok{(freq, }\DataTypeTok{decreasing =} \OtherTok{TRUE}\NormalTok{), }\DataTypeTok{col =} \StringTok{"pink"}\NormalTok{, }\DataTypeTok{las =} \DecValTok{2}\NormalTok{)}
\end{Highlighting}
\end{Shaded}

\begin{itemize}
\tightlist
\item
  Dans quelles proportions les termes de valeur d'association \(r\)
  inférieure à \(-0.1\) apparaissent-ils dans les documents ?
  Représenter graphiquement ces proportions à l'aide d'un diagramme en
  barre.
\end{itemize}

\begin{Shaded}
\begin{Highlighting}[]
\NormalTok{  i_have_a_dream}
  \KeywordTok{barplot}\NormalTok{(}\KeywordTok{sort}\NormalTok{(freq, }\DataTypeTok{decreasing =} \OtherTok{TRUE}\NormalTok{), }\DataTypeTok{col =} \StringTok{"palegreen"}\NormalTok{, }\DataTypeTok{las =} \DecValTok{2}\NormalTok{)}
\end{Highlighting}
\end{Shaded}

\paragraph{Modèles d'apprentissage}\label{modeles-dapprentissage}

\begin{itemize}
\item
  `A l'aide des outils de \emph{keras}, ajuster des modèles
  d'apprentissage aux données contenues dans le défi : ``x\_imbd'' et
  ``y\_imbd''. Considérer la graine du générateur aléatoire
  \texttt{seed} comme un hyper-paramètre (\texttt{set.seed(seed)}).
\item
  Dans un tableau, décrire les performances de 6 méthodes choisies pour
  des échantillons d'apprentissage et de test que vous aurez créés
  vous-mêmes à partir des données ``x\_imbd'' et ``y\_imbd''. Les
  performances seront mesurées par les erreurs de classification et
  d'entropie croisée (log loss).
\item
  Donner le code R correspondant au meilleur modèle que vous avez ajusté
  (chunk ci-dessous). On considèrera la graine du générateur aléatoire
  comme un hyperparamètre supplémentaire du modèle. Donner la valeur
  finale d'entropie croisée estimée sur votre ensemble test. On fera en
  sorte que le résulat soit complètement reproductible à partir du code
  proposé.
\end{itemize}

\begin{Shaded}
\begin{Highlighting}[]
\CommentTok{# code de mon meilleur modèle}
\CommentTok{# Ne pas oublier d'inclure la seed du générateur aléatoire }
\CommentTok{# set.seed(seed = 5427381)}
\NormalTok{  model <-}\StringTok{  }\NormalTok{deus_ex_machina}
\end{Highlighting}
\end{Shaded}

\section{Règles de rendu}\label{regles-de-rendu}

\begin{itemize}
\item
  Ce projet donne lieu à un compte rendu séparé, dans lequel on aura
  pris soin de reporter les sections de ce document comportant vos
  réponses, commentaires, et vos codes complétés. Merci de ne pas
  reporter les sections ``théoriques''.
\item
  Les codes R doivent être inclus dans le texte du compte-rendu (menu
  \textbf{Insert}) et l'ensemble doit être commenté avec précision.
  \textbf{Les commentaires compteront pour une part importante dans la
  note}.
\item
  Le compte-rendu doit être déposé \textbf{dans TEIDE}.
\item
  Le compte-rendu doit être déposé \textbf{au format HTML uniquement} et
  intitulé ``Rendu\_Projet\_1.html''. Utiliser la fonction
  \textbf{knitr} du menu de rstudio pour obtenir le document au format
  souhaité. \textbf{Les fichiers sources (Rmd ou nb.html) ne seront pas
  acceptés}.
\end{itemize}


\end{document}
